\chapter{Concetto di Cross Platform}\label{ch:crossplatform}

L'avanzare tecnologico e l'impronta sempre più grande che gli smartphone, o comunque i dispositivi tecnologici personali con cui interagiamo, hanno sulla nostra vita, ha reso lo sviluppo di applicativi un materia sempre più importante e remunerativa a livello globale, andando ad incidere su una fetta di mercato che riguarda quasi totalmente la popolazione mondiale.

L'approccio mobile-first la fa da padrone, in quanto ormai i computer sono stati perlopiù sostituiti da dispositivi che possiamo tenere sempre più a portata di mano come smartphones e smartwatches, e le aziende hanno sentito la necessità di spostare i loro canali di comunicazione favorendo lo sviluppo di app mobili in modo da rivolgersi ad un pubblico più vasto. E per la dura legge dell'economia si preferisce sempre pagare di meno per un progetto che è più facile da mantenere e più compatibile.
Con il concetto di cross-platform si intende infatti la capacità di sviluppare un'applicativo compilabile su diversi processori e sistemi operativi attraverso l'utilizzo di un singolo linguaggio, senza quindi creare due versioni logicamente equivalenti ma che hanno bisogno di essere scritte in linguaggi diversi per funzionare su dispositivi eterogenei.

Alcuni esempi di linguaggi cross-platform sono il C, Java, PHP, Python, ecc.. Questi infatti possono essere compilati su diverse macchine indipendentemente se su queste è stato installato un qualsiasi sistema operativo, rendendo quindi possibile scrivere codice che sarà chiaramente compatibile e condivisibile tra diversi utenti possessori di macchine diverse.

Il cross-platform è una tematica molto importante per quanto riguarda lo sviluppo di applicativi in quanto permette un risparmio di tempo considerevole in quanto non è necessario scrivere la stessa logica in linguaggi diversi, e di conseguenza anche un risparmio economico per le aziende, dovendo mantenere una singola versione di codice e non due sostanzialmente uguali come avverrebbe normalmente.

Sebbene non sia tutto oro quel che luccica in quanto uno sviluppo nativo garantisce comunque la massima espressività logica pelìr il dispositivo per cui stiamo sviluppando ed un supporto completo, la possibilità di avere un solo linguaggio per diverse piattaforme e essere capaci di creare UI identiche tra una piattaforma all'altra rende lo sviluppo cross-platform molto appetibile.

Nel nostro caso l'utilizzo di tecnologie cross-platform ci ha permesso di sviluppare una applicazione che ha la facoltà di essere compilata ed eseguita sia su device basati su Android sia IOS, nonchè possa essere convertita anche in una web app.In particolare abbiamo raggiunto il nostro scopo attraverso l'uso del framework open source Ionic e integrando Capacitor, runtime che supporta sia Swift per IOS, sia Java per Android. 

La nostra volontà è stata quella di adattare un approccio cross platform in modo tale da poter integrare la nostra web app su dispositivi mobile senza la necessità di dover implementare due diverse versioni di codice e senza il bisogno di imparare due differenti linguaggi per lo sviluppo nativo. Il risultato quindi è stato quello di una app che si modella perfettamente sia in ambiente mobile che desktop e che richiede un bassissimo livello di mantenimento, in quanto le future modifiche verranno eseguite solo su una versione di codice.

