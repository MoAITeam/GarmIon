\chapter{Scopo del progetto}\label{ch:scopo}

Lo scopo del progetto, dunque, è stato quindi quello di sviluppare un'applicativo nativo cross-platform che permettesse all'utente di avere una sorta di stilista digitale sempre a portata di mano. Quest'ultimo potrà infatti andare a creare il proprio armadio virtuale andando a scattare delle foto ai suoi capi di abbigliamento e, una volta aggiunti, ricercarli attraverso dei filtri.

Per quanto riguarda l'aggiunta del capo, oltre a dover inserire i dati principali del prodotto come colore, categoria e stagione, qui sotto spiegati, all'utente verrà inoltre richiesto di scattare una foto. Abbiamo pensato che l'utilizzo di un segmentatore potesse essere perfettamente votato alla causa: in questo modo l'indumento verrà completamente ritagliato dallo sfondo in modo tale da escludere eventuali altri oggetti, che oltre ad essere inutili da un punto di vista informativo potrebbero condurre in errore le reti neurali, le quali potrebbero considerarli in una eventuale raccomandazione.

I filtri che permettono di discriminare la ricerca in base a 3 diversi tag:
\begin{itemize}
\item Colore: quindi si va a filtrare tutti quegli abiti con il colore selezionato
\item Categoria: ovvero se l'indumento è un top o un bottom, ad esempio una TShirt è un top mentre un paio di jeans è un bottom.
\item Stagione: in base a quale stagione sarà più consono indossare il vestito selezionato.
\end{itemize}

Una volta aggiunti, i vestiti potranno dunque essere selezionati e l'applicazione andrà a proporre degli outfit che costituiscono un match , chiaramente in base alla correttezza di abbinamento dei colori. L'utente tuttavia avrà la facoltà, in questo caso, di poter ulteriormente filtrare i risultati in base al colore dell'indumento proposto, ed in base al mood in cui si trova e all'evento sociale per il quale utilizzerà tale outfit.

Chiaramente la raccomandazione in base a stile ed evento è adempiuta dalle due reti neurali che avevamo a disposizione: queste prenderanno inizialmente in ingresso l'indumento iniziale e proporranno una serie di abbinamenti coerenti. Ogni abbinamento verrà quindi scansionato per determinare il mood e l'evento appropriati, informazione che quindi verrà attaccata all'outfit stesso. A questo punto, una volta che l'utente deciderà di discriminare i risultati in base a questi filtri, basterà semplicemente controllare se il metadato corrispondente è coerente o meno con la cernita effettuata.

Una volta trovato l'abbinamento o gli abbinamenti l'utente dovrà essere in grado di salvarli in locale e poterli visualizzare con persistenza su un'altra tab, in modo da poter sfruttarli in più occasioni, anche qui potendo discernere in base allo stile e alla situazione sociale in cui sarebbe perfetto indossarli. Ovviamente dovrà essere in grado anche di visualizzare il mood e l'evento per cui erano stati pensati tali outfit, e inoltre dovrà poterli modificare e/o eliminarli.  
