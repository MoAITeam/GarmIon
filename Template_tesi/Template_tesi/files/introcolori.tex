\chapter{Teoria di Kobayashi}\label{ch:teoria}

Parliamo di bias ogni qualvolta una suggesione ci porta ad agire in modo diverso da quanto saremmo soliti fare se quella particolare inclinazione non ci avesse mai colpiti. A livello psicologico, è intuibile pensare che le emozioni che proviamo potrebbero essere proprio quella influenza che ci porta a comportarci in un modo o nell'altro. Per deduzione, anche il nostro modo di vestire quotidiano e i colori indossati potrebbero far parte di quelle azioni che le varie suggestioni emotive ci portano a compiere.

Difatti per quanto riguarda la teoria filosofico-scientifica che fa da fondamento alla rete neurale che abbiamo utilizzato per suggerire l'outfit in base al mood in questo progetto è derivante dagli studi sul colore e dei suoi accostamenti dello studioso giapponese Shigenobu Kobayashi. Nel libro "Color Image Scale" edito nel 1990 oltre a studiare le possibili correlazioni con vastissimi ambiti come forme e cucina, Kobayashi si sofferma sul tema forse più filosofico,che riprende maggiormente le correnti di pensiero del Sol Levante, la sua madre patria : le emozioni.

Nel senso più letterale del termine Kobayashi associa a dei gruppi di colore una particolare emozione, dopo aver chiaramente analizzato il problema da un punto di vista psico-sociologico. Applicando i suoi studi alla moda, è intuitivo pensare che se una persona si veste usando colori prettamente scuri e cupi potrebbe esprimere una sensazione negativa mentre colori sgargianti  e luminosi potrebbero trasmettere benestare. Di fatto lo studioso giapponese va a formalizzare queste intuizioni creando così un sistema biunivoco che correla una particolare emozione ad una combinazione di colori.

Come spiegheremo più avanti, nella sezione dedicata ai modelli usati per la categorizzazione, le triplette di colore identificate da Kobayashi vengono fatte coincidere con precisi stili che riflettono l'attitudine di una persona in modo da suggerire un abbinamento corretto per ogni occasione
